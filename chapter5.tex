\chapter{Discussion}

With the growing need to analyze microclimates of cities in order to protect them against new phenomena like UHIs, this paper proposes the use of citizen-owned sensor networks that offer a higher spatial and temporal resolution of data points in comparison to traditional approaches such as relying on LST data. This approach also comes with challenges such as observing areas without stationary sensors and identifying poor quality data-points from broken or incorrectly installed sensors. With the obtained data, we create a temperature interpolation service that predicts temperature data between data points using a regression model that can act as a building block for further temperature-based analysis by abstracting the underlying single data-points in the data-layer away.\\
In order to improve the quality of temperature predictions for unobserved areas, we investigate how temporary sensor readings, f.e. by attaching sensors to buses, bikes or e-scooters, can be used to capture temporary local meteorological snapshots, and how these snapshots can be incorporated into the interpolation process. We also evaluate which features need to be captured to generate the most accurate predictions, if machine-learning is a suitable approach, and how it compares to more traditional geostatistical approaches.