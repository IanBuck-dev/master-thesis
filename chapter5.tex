\chapter{Preparation of Datasets}
\label{chap:preparations data sets}
% Todo: 6 pages?

Next to the ML model selected, the data used to train and evaluate the model has a major influence on its performance. If the data is not representative or information is missing about the underlying process that the model should be fitted to, there can be errors, bias or an inability to generalize well to new data. In this chapter, we take a look at potential features for air temperature, at data sources for these features, and discuss the construction process of the datasets used in this work.\\
% Overall dataset availability
In the field of natural language processing (NPL) and computer vision (CV), there is an abundance of large available datasets which have a big contribution to the advancements in the field, like annotated datasets as provided by Google Research~\footnote{\url{https://research.google/resources/datasets/}, \textit{last accessed: 05.08.2023}}. In comparison in the field of climate research, there are also many datasets, including satellite data, weather station data, and climate model data, however they are highly distrubuted or often not openly available. Platforms such as Google Earth Engine~\cite{gorelick2017google} try to address this issue, however the dataset catalog~\footnote{\url{https://developers.google.com/earth-engine/datasets}, \textit{last accessed: 05.08.2023}} is still limited and does not include datasets offered by local authorities or other research institutions, such as universities.\\
% How does the perfect dataset for air temperature look like?
For the specific use case of temperature interpolation in urban areas, an optimal dataset would contain high spatial and temporal resolution sensor data, e.g.\ a high sensor density and a low time interval of f.e.\ five to ten minutes. Additionally, the sensor placement and sensor quality have a high influence on the accuracy of the sensor readings~\cite{oke2006guideline}, therefore the correct placement and calibration of the sensors needs to be guaranteed. Such requirements are not met by traditional weather station networks as the spatial coverage is too low, as a single weather station is not enough to capture the urban microclimate~\cite{oke2017urban}. The weather station locations of the DWD are shown in figure~\ref{fig:dwd sensor locations germany}, which shows that usually at most one weather station is available per city. The weather stations however offer a high temporal resolution in addition to very high quality sensors, which is why they can be used as refrence stations for quality control of other sensors, as discussed in section~\ref{sec:quality control}.\\
A solution to the problem of low spatial coverage is the usage of sensor networks. There are several projects that run dense urban-climate monitoring networks~\cite{muller2013sensors} such as the Helsinki Testbed~\cite{koskinen2011helsinki} or the Birmingham Urban Climate Laboratory~\cite{warren2016birmingham}, however access to those datasets is limited, for example due to outdated links or the need to request access. Due to the high cost of running such dense sensor networks with high sensor and maintenance costs, professionally run sensor networks are rare and are often only run for a limited time period until project funds run out.\\
As an alternative, sensor networks can also be crowdsourced and run by citizens, distributing the cost of individual sensors as well as maintenance costs among many. Especially with advances in sensor technologies, lost cost and compact sensors are more affordable than ever while still providing good data quality~\cite{grimmond2006progress, rundel2009environmental}. In the context of meterological data, such such sensor networks are often referred to as citizen weather station (CWS)~\cite{meier2017crowdsourcing} or personal weather station (PWS)~\cite{hahn2022observations} networks. In this work, we use the term PWS.\\
The main downside of this approach is the lack of quality control and meta data, as the sensors are usually placed by non-professionals in suboptimal locations, e.g.\ in direct sunlight or too close to walls, leading to incorrect readings or bias in the data. A lack in meta data can also lead to issues, if for example exact positions of the sensors are not known and information about the height of the sensor above ground is missing. However, other concerns such as data privacy also need to be accounted for, as such weather stations are often placed on private property.\\
In this work, data from PWS networks is used to create datasets for the training and evaluation of ML models for air temperature interpolation. In the following sections, we take a look at available PWS providers and their data, look at potential features for air temperature interpolation, discuss additonal pre-processing steps such as quality control or sensor height correction, and finally discuss the construction of the datasets used in this work.

\section{Private Weather Station Network Providers}

PWS providers offer a platform for users to upload their sensor data and either sell weather stations and sensors themselves such as Netatmo, or provide guides to allow users to connect their own sensors to the platform such as Sensor.Community. Netatmo data in particular has been used in several studies~\cite{meier2017crowdsourcing, hahn2022observations, venter2020hyperlocal, zumwald2021mapping} and has seen complementary studies for example discussing QC processes~\cite{fenner2021crowdqc+}, later seen in Section~\ref{sec:quality control}. There are also other PWS network providers such as Sensor.Community or Weather Underground (WOW) that have been used in several studies~\cite{ho2014mapping}. In order to find out which provider best fits our needs in this work, the following section compares the different providers and their data.

\subsection{Sensor.Community}

Sensor Community~\footnote{\url{https://sensor.community/en/}, \textit{last accessed: 05.08.2023}} is a contributers driven global sensor community that creates Open Environmental Data, and has an archive~\footnote{\url{https://archive.sensor.community/}, \textit{last accessed: 05.08.2023}} of their historical sensor data world wide. There are no quality measures recorded for each sensor, but as crowd-sourced sensor data tends to have a lower quality than professsionally setup sensors, e.g.\ sensor placement by non-professionals, we need to explore how the data quality looks like.
In Figure~\ref{fig:temperature_sensor_community_map}, where we see the greater Hamburg area with a currently reported temperature of 25°C by the DWD Fuhlsbüttel station, there are multiple sensors that report a temperature of 30°C and above, which could be either due to the sensor being placed in direct sunlight or due to the sensor being faulty. An outlier near Pinneberg is shown in Figure~\ref{fig:temperature_sensor_community_outlier}, where one sensor reports 25°C, as currently expected, and one sensor reporting 50°C, which is clearly an outlier. This data quality issue needs to be adressed in the data pre-processing step and can result in a significant reduction of available data. This was also an issue dicussed in~\cite{meier2017crowdsourcing}, as ``erroneous metadata, failure of data collection, and unsuitable exposure of sensors lead to a reduction of data availability by 53 \%``.
From a meta-data perspective, there is no information on the sensor height above ground as well as no quality measures for each sensor, or information on the sensor location accuracy.

\begin{figure}[ht]
    \centering
    \includegraphics[width=1\textwidth]{images/sensor_community_temperature_map.png}
    \caption{Temperature map from Sensor Community for Hamburg, Germany, on 22.06.2023 12:51h with the DWD reference at 25°C}
    \label{fig:temperature_sensor_community_map}

    \includegraphics[width=0.8\textwidth]{images/sensor_community_outliers.png}
    \caption{Temperature outlier from Sensor Community for Hamburg, Germany, on 22.06.2023 12:51h with the DWD reference at 25°C}
    \label{fig:temperature_sensor_community_outlier}
\end{figure}

Overall, there are around 11.738 active sensors~\footnote{as of 24.06.2023}. Of these sensors, many are located in Germany, as seen in Appendix~\ref{sensor_community_sensors_by_countries}, and almost half of them are of type BME 280, which is a lost-cost Bosch sensor which can measure temperature, pressure and humidity. The sensor locations as of may 2023 are shown in~\ref{fig:sensor community sensor locations germany}. DHT22 sensors can measure temperature and humidity, BMP280 and BMP180 sensors can measure temperature and pressure, and BME280 sensors can measure temperature, pressure and humidity.\\

\begin{figure}[ht]
    \centering
    \includegraphics[width=1\textwidth]{images/sc_sensor_locations_germany.png}
    \caption{Sensor locations of Sensor Community in Germany, as of 01.05.2023, of sensor type DHT22 (2590 sensors), BME280 (1558 sensors), BMP280 (100 sensors), BMP180 (72 sensors)}
    \label{fig:sensor community sensor locations germany}
\end{figure}

\subsection{Netatmo}

Netatmo~\footnote{\url{https://www.netatmo.com/en-eu}} is a French company that sells smart-home devices including outdoor weather stations, indoor sensors for air quality, as well as other products such as smart cameras. They host a weather map~\footnote{\url{https://weathermap.netatmo.com/}, \textit{last accessed: 06.08.2023}} where customers can share their outdoor weather station data. They provide an API to access current weather station data as well historic data from individual outddor sensors and modules.
They provide their historical and current weather data for commercial partners or partners in the research and education sector. They are part of the EUMETNET project~\footnote{\url{https://www.eumetnet.eu/}, \textit{last accessed: 06.08.2023}} which is a network of 31 European meteological and hydrological services (NMHSs). The project aims to facilitate the exchange of weather data and to improve the quality of weather forecasts, especially in the context of PWS~\cite{hahn2022observations}. There are currently no openly historical datasets available from Netatmo data, only private datasets~\footnote{\url{https://catalogue.ceda.ac.uk/uuid/e8793d74a651426692faa100e3b2acd3}, \textit{last accessed: 06.08.2023}} that are only available for partners such as EUMETNET members. They offer an educational program~\footnote{\url{https://www.netatmo.com/en-eu/weather-with-netatmo}} to acess temporally and spatially limited amounts of data that is usually only available to commercial partners.

% precisions
% Please add the following required packages to your document preamble:
% \usepackage{booktabs}
\begin{table}[]
\begin{tabular}{@{}lllll@{}}
\toprule
Measurement    & Unit    & Measurement Range      & Precision  & Recording Frequency              \\ \midrule
Temperature    & °C      & -40°C to 65°C          & 0.3°C   & averaged over 5 min              \\
Humidity       & \% (RH) & 0 to 100\%             & 3\%     & -                                \\
Air Pressure   & mbar    & 260 to 1160 mbar       & 1mbar   & -                                \\
Noise          & dB      & 35 to 120 dB           & -       & -                                \\
Wind Speed     & m/s     & 0 to 45 m/s (160 km/h) & 0.5 m/s & every 6 sec, averaged over 5 min \\
Wind Direction & °       & 0 to 359°              & 5°      & every 6 sec, averaged over 5 min \\
Rainfall       & mm/h    & 0.2 to 150 mm/h        & 1mm/h   & every 5 min (bucket is emptied)  \\ \bottomrule
\end{tabular}
\caption{Netatmo Sensor Specifications (Vendor reported)}
\label{tab: netatmo sensor specs}
% todo: adjust to page size, maybe move to dataset chapter
\end{table}

In the context of collecting meteological data, the smart weather products are of particular interest. These include a smart outdoor weather station that collects air temperature, humidity and air pressure, an anemometer that collects wind speed and direction, and a rain gauge. The sensor specifications, as reported by the vendor himself, is reported in Table~\ref{tab: netatmo sensor specs}.\\
In this work, data from Netatmo stations is used as Netatmo offers a large amount of sensors in Germany in urban areas, exemplified by Figure~\ref{fig:netatmo sensor locations hamburg} for the region of Hamburg, and by Figure~\ref{fig:netatmo sensor locations stuttgart} for the region of Stuttgart.
The developer portal~\footnote{\url{https://dev.netatmo.com/apidocumentation}} offers a way to programmatically access all public sensor measurements via a REST API, however each request has a limit on the spatial extend of the requested area for the current weather data. For historic data, the limit per request per sensor is 1024 data points. The API has a tight rate limit per application. For applications below 100 users, the rate limit is 2000 requests every hour and 200 requests every 10 seconds across all users, and 500 requests every hours and 50 requests every 10 seconds per user~\footnote{\url{https://dev.netatmo.com/guideline\#rate-limits}, \textit{last accessed: 06.08.2023}}. In this work, we use the REST API to collect sensor data from Netatmo sensors.

\begin{figure}[htp]
    \centering
    \includegraphics[width=1\textwidth]{images/netatmo_sensor_locations_hamburg.png}
    \caption{Sensor locations of Netatmo in Hamburg, Germany, as of 28.06.2023}
    \label{fig:netatmo sensor locations hamburg}

    \includegraphics[width=1\textwidth]{images/netatmo_sensor_locations_stuttgart.png}
    \caption{Sensor locations of Netatmo in Stuttgart, Germany, as of 19.06.2023}
    \label{fig:netatmo sensor locations stuttgart}
\end{figure}

\subsubsection{Quality Considerations}

Netatmo sensors have a good measurement accuracy, however due to the compact design, an aluminium housing, poor ventilation due to the small case, no dedicated radiation screen resultiing in a proneness to radiative errors, and therefore overall slow sensor-response time~\cite{meier2017crowdsourcing, buchau2018modelling}, Netatmo weather stations have a systematic bias that influences data quality. Due to the uniformity of Netatmo sensors, e.g.\ all sensors are build in the same way, this bias could be corrected in the QC step, however this is not further explored in this work.

\subsection{Other Providers}

Other sources for crowdsourced weather station data include WeatherObservationsWebsite (WOW)~\footnote{\url{https://wow.metoffice.gov.uk/}} and Weather Underground~\footnote{\url{https://www.wunderground.com/}}.
WOW is a platform run by the UK Met Office, which is the UK's national weather service, and has a dense sensor coverage in the UK and the Netherlands as seen in figure~\ref{fig:wow sensor locations}.
Weather Underground is a commercial weather service which also provides a crowdsourced weather station network. Unfortunately, Weather Underground only provides an API for users with a registered weather station or other bulk download options for historical data. The website would allow for manual download of historical data, but this is not feasible for the amount of data needed for this work.

\begin{figure}[ht]
    \centering
    \includegraphics[width=1\textwidth]{images/wow_sensor_locations.png}
    \caption{Temperature sensor locations from WOW, acessed on 05.07.2023}
    \label{fig:wow sensor locations}
\end{figure}

\section{Reference Data Providers}

In order to add additional validation to crowdsourced weather data, reference data from (official) weather stations can be used. These weather stations should be setup according to current World Meterological Organization (WMO) guidelines~\cite{wmo2018guide} in order to ensure high data quality. These standards are either achived by offical weather services, or by institutions such as universities whose sensors are maintained by experts.

\subsection{DWD}

The official german weather service (DWD) has many objectives, that are defined by the DWD-law in Germany. Its tasks include meterological and climatological monitoring of the atmosphere, meterologically securing the airspace for civil aviation, monitoring the maritim climate, and more. The DWD operates a large monitoring network and publishes most of it's data via its OpenData portal~\footnote{\url{https://opendata.dwd.de/}, last accessed 13.07.2023}.\\
The main advantages of the DWD data are high data quality through reference instruments and proper setup according to WMO guidelines~\cite{wmo2018guide}. The main disadvantage is the low spatial coverage of the data, as stations are sparsely distributed to measure the overall mesoscale climate, as seen in Figure~\ref{fig:dwd sensor locations germany}. Additionally, a lot of the public weather stations are located close to airports, which are usually located outside of cities, and therefore not suitable for measuring urban microclimates.

\begin{figure}[ht]
    \centering
    \includegraphics[width=1\textwidth]{images/dwd_weather_station_locations_germany.png}
    \caption{DWD Weather Station Locations in Germany, \url{https://opendata.dwd.de/climate_environment/CDC/observations_germany/climate/subdaily/standard_format/KL_Standardformate_Beschreibung_Stationen.txt}, accessed 28.06.2023}
    \label{fig:dwd sensor locations germany}
\end{figure}

\subsubsection{Urban Climate Stations}

Next to the official weather stations, the DWD also operates urban weather stations, however there are currently only four stations in the following cities:

\begin{itemize}
    \item Berlin-Alexanderplatz, Berlin, Berlin
    \item Freiburg-Mitte, Freiburg, Baden-Württemberg
    \item Hannover-Nordstadt, Hannover, Niedersachsen
    \item Dresden-Neustadt, Dresden, Sachsen
\end{itemize}

The number of urban weather stations is planned to be gradually extended to reach 10 stations with the locations being primarily determined by the measurement objectives such as determining a city's maximum UHI intensity~\footnote{\url{https://www.dwd.de/EN/climate\_environment/climateresearch/climate\_impact/urbanism/urban\_heat\_island/urbanheatisland\_node.html}, last accessed 12.07.2023}. Due to the low number of weather stations, their data is not used in this work.

\subsubsection{Weather Radar}

The DWD also operates a network of weather radars~\footnote{\url{https://www.dwd.de/DE/leistungen/radarprodukte/radarprodukte.html}, \textit{last accessed 12.07.2023, not available in english}} that are used to measure precipitation and wind speed. This data could be interesting in the context of air interpolation in order to detect precipitation events that have a mayor influence on humidity and temperature or to detect wind speeds that also play an important factor in dissapating heat and transporting it away from urban areas.
Due to the limited scope of this work, this data is currently not used.

\subsection{Locally Operated Weather Stations}

Next to offical weather services, many public and private institutions operate weather stations. The following section lists two examples of such institutions, the Meterological Institute of the University of Hamburg and the Office for Environmental Protection of the City of Stuttgart.

\subsubsection{University of Hamburg – Meterological Institute}

- address local climate concerns and support decision making
- Projekt HUSCO (Hamburg Urban Soil Climate Observatory)

\subsubsection{City of Stuttgart – Office for Environmental Protection}

The Office for Environmental Protection of the city of Stuttgart has the goal of monitoring the climate in Stuttgart and the surrounding area and improve the living conditions of the citizens. The main focus of the office is on air quality, noise, and (urban) climate. It operates several weather stations in the city of Stuttgart, which are shown in Figure~\ref{fig:stuttgart weather station locations}.

% TODO: figure

\section{Quality Control}
\label{sec:quality control}

Quality control (QC) is an essential step in the process of data analysis and preparation. The goal is to identify and remove outliers in the data that are due to placement errors of sensors, sensor malfunctions, sensor inaccuracies or other errors. In the context of PWS, weather stations are placed and maintained by non-professionals, making QC even more important. One of the main challenges in the context of (hyper-) local urban air temperature data is to not flag data as outliers that is representative of the local climate in case of extreme temperature, e.g.\ heat islands, and at the same time identify erronous or wrongly placed sensors, e.g.\ too close to walls, in direct sunlight, indoors, etc. Additionally, current PWS networks do not track sufficient metadata on the sensor placement, e.g.\ sensor height, which also plays an important role in protecting the privacy of citizens and not exposing too accurate sensor locations.\\
Due to the the popularity of Netatmo weather station data in research due to high spatio-temporal resolution, there are several software libraries available that help simplify and automate the QC process. These tools were primarily developed for Netatmo temperature data, however CrowdQC and TITAN can also be used for other nearly-normally distributed data sources~\cite{hahn2022observations}. The following tools are available:

\begin{itemize}
    \item CrowdQC (R package~\footnote{\url{https://doi.org/10.14279/depositonce-6740.3}})
    \item CrowdQC+~\cite{fenner2021crowdqc+} (R package~\footnote{\url{https://github.com/dafenner/CrowdQCplus}})
    \item TITAN (R package~\footnote{\url{https://github.com/metno/TITAN}})
    \item NetatmoQC (Python 3 package~\footnote{\url{https://source.coderefinery.org/iOBS/wp2/task-2-3/netatmoqc}})
\end{itemize}

In this work, CrowdQC+ is used for QC as it offers improvements and bug fixes compared to CrowdQC. It's an open-source software library written in R, a popular programming language for statistical applications. The data needs to be in the following format:

\begin{itemize}
    \item \textit{p\_id}: The unique ID of the station
    \item \textit{time}: The time of the measurement
    \item \textit{ta}: The air temperature in degree Celsius
    \item \textit{lon}: The longitude of the station
    \item \textit{lat}: The latitude of the station
    \item \textit{z}: The height of the station in meters, optional
\end{itemize}

The CrowdQC+ library implements the following required steps of QC:\ Metadata Check, Distribution Check, Data Validity, Temporal Correlation, Spatial Buddy Check. There are also the following optional steps available, that are currently not used: Temporal Interpolation, Daily Validity, Validity in Time Period, and Correction for Time Constant.
The steps used in this work are shown and explained in Table~\ref{tab: qc_steps}, including the number of data and stations available after each step.\\
In their own study, CrowdQC+ kept 47.1\% and 69.2\% of data after steps m1-5, and only 20.7\% and 29.5\% after steps o1-o3, for the cities Amsterdam and Toulouse respectively~\cite{fenner2021crowdqc+}, given default parameters. In that setting, CrowdQC kept more data with 41.0\% in Amsterdam and 54.9\% in Toulouse. In this work, CrowdQC+ is used with default parameters excluding height validation, as this data was not available for almost all sensors. Additionally, only the first 5 required steps are used, as the optional steps are not needed for the interpolation. The input data for CrowdQC+ also needs have the same temporal resolution and intervals.\\
In this study, we use a 10 min interval to have a high temporal resolution and use the default parameters except excluding the heigh check due to the missing values. Important to note here, that in the following, only the air temperature is validated and not other measurements such as pressure or humidity. CrowdQC+ could be used for other approximately normally distributed features, however there hasn't been more specific research in this direction. We assume that a station that seems to be setup correctly and produces good air temperature measurements, also captures the other measurements correctly for simplicity reasons.

\begin{figure}[htp]
    \centering
    \includegraphics[width=1\textwidth]{images/sensor_community_qc_january_23.png}
    \caption{QC Results for Sensor.Community Data for Germany, January 2023}
    \label{fig:qc sensor community jan 23}

    \centering
    \includegraphics[width=1\textwidth]{images/sensor_community_qc_june_23.png}
    \caption{QC Results for Sensor.Community Data for Germany, June 2023}
    \label{fig:qc sensor community june 23}
\end{figure}

\subsection{Quality Control for Sensor.Community}

For Sensor.Community, we can see several interesting things for the air temperature. The first is, that in january 2023 less data is lost due to QC compared to June 2023. This could be to the fact, that in colder environments with less solar radiation, sensor placement, f.e. close to buildings, has less of an influence. In comparison, june 2023 had many hotter days, therefore it could be that more extreme readings are flagged as outliers. We can also see that Sensor.Community loses a lot of stations in step m4 in June 2023 compared to January 2023, which is the temporal correlation with the median of all stations. This could also be due to a higher temperature difference accross Germany, therefore comparing smaller areas could be beneficial for this. We can also see, that the m5 check, which is the buddy check with surrounding stations, also removes a lot of stations which could be due to the low station density.\\
The CrowdQC+ library can theoretically be used to validate other near-normally distributed variables, however this could change the way the parameters should be set. Due to the limited scope, for other readings, e.g. relative humidity and atmospheric pressure, we simply remove default values. As an improvement, for other variables a more sophisticated QC process should be used. In additon, due to the low sensor density and the fact, that all types of sensors used are good low cost sensors, we simply combine all sensor readings after QC step m5 into one dataset and ignore the sensor type.\\
After the QC process, the Sensor.Community sensor locations left are shown in~\ref{fig:qc sensor community germany june 23}. In this figure we can see, that there are many sensors left in Stuttgart, Hamburg, Munich, and some in Cologne. Due to DWD stations only being present in Hamburg and Stuttgart, those two areas are candidates for further usage.

\begin{figure}[ht]
    \centering
    \includegraphics[width=1\textwidth]{images/sensor_community_locations_germany_after_qc_june_23.png}
    \caption{QC Results for Sensor.Community for Germany, June 2023}
    \label{fig:qc sensor community germany june 23}
\end{figure}

% TODO: Update table with final data
\begin{sidewaystable}
\label{tab: qc_steps}
\caption{Quality Control Steps of CrowdQC+}
\resizebox{\textwidth}{!}{
\begin{tabular}{lllllll}
\hline
\textbf{}         & \textbf{Id} & \textbf{Name of Step}        & \textbf{Functionality}                                                                                                                                                                                                                                                        & \textbf{\% of Data} & \textbf{Num Stations} & \textbf{Num Values} \\ \hline
\textbf{Required} &             &                              &                                                                                                                                                                                                                                                                               &                                                                                    &                                                                                         &                                                                  \\
                  & m1          & Metadata Check               & \begin{tabular}[c]{@{}l@{}}Validates longitude and latitude values and removes stations with\\ identical values. Mainly aims to remove stations with default values\\ from locations from IP addresses due to improper configuration\\ by the end-user\end{tabular}           & 97.40\%                                                                            & 1077                                                                                    & 2.082.283                                                        \\
                  & m2          & Distribution Check           & \begin{tabular}[c]{@{}l@{}}Primarily targets radiative error that lead to unrealistic high ta\\ values and sensors installed indoors\end{tabular}                                                                                                                             & 86.50\%                                                                            & 1041                                                                                    & 1.849.247                                                        \\
                  & m3          & Data Validity                & \begin{tabular}[c]{@{}l@{}}Checks values of stations that did not pass m2. If more than 20\%\\ of data didn't pass the check, the station is considered to be faulty\\ and is removed\end{tabular}                                                                            & 85.20\%                                                                            & 845                                                                                     & 1.821.479                                                        \\
                  & m4          & Temporal Correlation         & \begin{tabular}[c]{@{}l@{}}Checks the temporal correlation between each station and the\\ median of all stations for a specified period of time, default 1\\ month. Targets indoor stations that have weak temporal\\ correlation to the median of all stations.\end{tabular} & 79.90\%                                                                            & 829                                                                                     & 1.708.061                                                        \\
                  & m5          & Spatial Buddy Check          & \begin{tabular}[c]{@{}l@{}}Neighbourhood-based check to identify outliers within a specific\\ area. Primarily targets radiation errors with too high ta values.\\ Defaults to radius of 3000m and 5 neighbours.\end{tabular}                                                  & 31.53\%                                                                            & 466                                                                                     & 674.004                                                          \\
\textbf{Optional} &             &                              &                                                                                                                                                                                                                                                                               &                                                                                    &                                                                                         &                                                                  \\
                  & o1          & Temporal Interpolation       & \begin{tabular}[c]{@{}l@{}}Step to interpolate missing values in the time-series of each\\ station to increase data availability\end{tabular}                                                                                                                                 & -                                                                                  & -                                                                                       & -                                                                \\
                  & o2          & Daily Validity               & Verifies robust calculations of daily values                                                                                                                                                                                                                                  & -                                                                                  & -                                                                                       & -                                                                \\
                  & o3          & Validity in Time Period      & Checks if enough values are available in a given time frame                                                                                                                                                                                                                   & -                                                                                  & -                                                                                       & -                                                                \\
                  & o4          & Correction for Time Constant & \begin{tabular}[c]{@{}l@{}}Sensors have different times that they respond to ta changes.\\ Due to Netatmo design flaws, a\\ constant correction for all stations can be applied.\end{tabular}                                                                                 & -                                                                                  & -                                                                                       & -                                                                \\ \hline
\end{tabular}}
\vspace{1ex}

{\raggedright This table shows the QC steps used in this work from the CrowdQC+ library, including the \% of data available after each step, the number of stations available and the number of values left after each step. Optional steps are currently not used. \par}
\end{sidewaystable}

\subsection{Quality Control for Netatmo}

\begin{figure}[htp]
    \centering
    \includegraphics[width=1\textwidth]{images/netatmo_qc_june_23.png}
    \caption{QC Results for Netatmo Data for Hamburg, June 2023}
    \label{fig:qc netatmo june 23}
\end{figure}

\section{Feature Engineering}
\label{sec:feature_engineering}

The goal of feature engineering is to create features from the available data that can be used as input for the machine learning models. Based on the features, different models can be used for completely different tasks such as interpolation compared to extrapolation. The process includes the selection of features, the extraction of features from the raw data, and the transformation of features into a format that can be used by the machine learning models.
Especially in the context of air temperature interpolation, a lot of domain knowledge is required to select the right features and model correlations between them correctly. The target feature in this work is the air temperature at canopy height, e.g. 2m height. The input features are a combination of sensor readings from weather stations, sensor networks such as Sensor Community and Netatmo, satellite data such as land cover and vegetation health or LST, and additonal meta data such as soil conditions or zoning plans. The goal of this section is to give an overview of the different features that can be used for air temperature interpolation and discuss several highly important features that are especially relevant in the context of urban microclimate.

\subsection{Feature Overview}

\subsubsection{Essential Climate Variables}

Essential Climate Variables (ECV) are a list of currently 50 variables that are proposed by the WMO to measure climate and climate change. The WMO regularily publishes updates on which climate variables to use and how to measure them~\cite{wmo2018guide}. ECVs are generally more focused on measuring climate on a global scale, however they also contain many variables that are relevant for urban microclimate, such as air temperature and land cover. There are three categories of ECVs, namely Atmosphere, Land, and Ocean. And overview of the ECVs can be found online~\footnote{\url{https://gcos.wmo.int/en/essential-climate-variables/table}, \textit{last accessed: 08.08.2023}}.\\

Hamburg ICDC overview:
- atmospheric data
    - air temperature
    - pressure
    - wind
    - reciptation
    - clouds
    - aerosols
    - humidity
    - radiation
    - climate indices
- ocean
    - water temperature
    - wave height (SSH)
    - salt content
    - tide
    - ocean color (e.g. plankton etc.)
    - climatology
    - ocean currents
- ice/snow
    - sea ice coverage
    - sea ice thickness
    - sea ice type
    - snow thickness (ice)
    - snow water equivalent (SWE)
    - land snow cover
    - glacier thickness
    - melting ponds
- land
    - albedo
    - surface temperature
    - vegetation
    - soil moisture
    - topography
    - short-wave radiation
    - perma-frost
- society
    - social science parameters

\subsubsection{Features used in Related Work}

Related studies can give a good overview of which features work best for air temperature interpolation. The used features can be roughly divided into three categories: in-situ measurements, satellite data, and additional meta data.
An important way to use satellite data is to calculate indexes out of the raw data. Alonso and Renard~\cite{alonso2020new} used among other data the indexes shown in Table~\ref{tab:alonso_indexes} to predict air temperature.

% Please add the following required packages to your document preamble:
% \usepackage[table,xcdraw]{xcolor}
% If you use beamer only pass "xcolor=table" option, i.e. \documentclass[xcolor=table]{beamer}
\begin{table}[ht]
\footnotesize
\centering
\begin{tabular}{|lll|lll|}
\hline
\textbf{}                & \textbf{Variables (Units)}                                                                   & \multicolumn{1}{c}{\textbf{\begin{tabular}[c]{@{}c@{}}Acquisition\\ Source\end{tabular}}} & \textbf{}                & \textbf{Variables (Units)}                                                            & \multicolumn{1}{c|}{\textbf{\begin{tabular}[c]{@{}c@{}}Acquisition\\ Source\end{tabular}}} \\ \hline
\rowcolor[HTML]{EFEFEF} 
                         & \textbf{Vegetation Index}                                                                    &                                                                                           &                          & \textbf{Radiation Index}                                                              &                                                                                            \\
                         & \begin{tabular}[c]{@{}l@{}}Normalized Difference Vegetation\\ Index (NDVI)\end{tabular}      & Landsat 8                                                                                 &                          & Spectral Radiance                                                                     & Landsat 8                                                                                  \\
                         & Enhanced Vegetation Index (EVI)                                                              & Landsat 8                                                                                 &                          & Emissivity                                                                            & Landsat 8                                                                                  \\
                         & \begin{tabular}[c]{@{}l@{}}Soil Adjusted Vegetation Index\\ (SAVI)\end{tabular}              & Landsat 8                                                                                 &                          & \begin{tabular}[c]{@{}l@{}}Tasseled Cap Transformation\\ Brightness\end{tabular}      & Landsat 8                                                                                  \\
                         & \begin{tabular}[c]{@{}l@{}}Tasseled Cap Transformation\\ Greenness (GVI)\end{tabular}        & Landsat 8                                                                                 & \cellcolor[HTML]{EFEFEF} & \cellcolor[HTML]{EFEFEF}\textbf{Building Index}                                       & \cellcolor[HTML]{EFEFEF}                                                                   \\
                         & Density of Low Vegetation                                                                    & LiDAR                                                                                     &                          & \begin{tabular}[c]{@{}l@{}}Normalized Difference Buit-Up\\ Index (NDBI)\end{tabular}  & Landsat 8                                                                                  \\
                         & Density of Medium Vegetation                                                                 & LiDAR                                                                                     &                          & Urban Index (UI)                                                                      & Landsat 8                                                                                  \\
                         & Density of High Vegetation                                                                   & LiDAR                                                                                     &                          & Index-based Built-Up Index (IBI)                                                      & Landsat 8                                                                                  \\
\cellcolor[HTML]{EFEFEF} & \cellcolor[HTML]{EFEFEF}\textbf{Water Presence Index}                                        & \cellcolor[HTML]{EFEFEF}                                                                  &                          & Building Density                                                                      & LiDAR                                                                                      \\
                         & \begin{tabular}[c]{@{}l@{}}Modified Normalized Difference\\ Water Index (MNDWI)\end{tabular} & Landsat 8                                                                                 & \cellcolor[HTML]{EFEFEF} & \cellcolor[HTML]{EFEFEF}\textbf{Urban Morphology}                                     & \cellcolor[HTML]{EFEFEF}                                                                   \\
                         & \begin{tabular}[c]{@{}l@{}}Normalized Difference Water\\ Index (NDWI)\end{tabular}           & Landsat 8                                                                                 &                          & Sky View Factor                                                                       & LiDAR                                                                                      \\
\cellcolor[HTML]{EFEFEF} & \cellcolor[HTML]{EFEFEF}\textbf{Bare Soil Index}                                             & \cellcolor[HTML]{EFEFEF}                                                                  &                          & \begin{tabular}[c]{@{}l@{}}Standard Deviation (STD) of\\ Building Height\end{tabular} & \begin{tabular}[c]{@{}l@{}}Local\\ Authority\end{tabular}                                  \\
                         & \begin{tabular}[c]{@{}l@{}}Normalized Difference Bareness\\ Index (NDBaI)\end{tabular}       & Landsat 8                                                                                 & \cellcolor[HTML]{EFEFEF} & \cellcolor[HTML]{EFEFEF}\textbf{Moisture Index}                                       & \cellcolor[HTML]{EFEFEF}                                                                   \\
                         & Bare Soil Index (BI)                                                                         & Landsat 8                                                                                 &                          & \begin{tabular}[c]{@{}l@{}}Tasseled Cap Transformation\\ Index\end{tabular}           & Landsat 8                                                                                  \\
                         & \begin{tabular}[c]{@{}l@{}}Enhanced Built-Up and Bareness\\ Index (EBBI)\end{tabular}        & Landsat 8                                                                                 &                          & \begin{tabular}[c]{@{}l@{}}Normalized Difference Moisture\\ Index (NDMI)\end{tabular} & Landsat 8                                                                                  \\
                         & Density of Bare Soil                                                                         & LiDAR                                                                                     &                          &                                                                                       &                                                                                            \\ \hline
\end{tabular}

\caption{Indexes used by Alonso and Renard~\cite{alonso2020new} to predict air temperature.}
\label{tab:alonso_indexes}
\end{table}

These indexes are either available as precalculated datasets, or can be calculated from raw satellite data. Especially the Google Earth Engine platform provides a lot of precalculated datasets from various sources such as MODIS~\cite{didan2021modis}, however each index is separately available, therefore requiring a lot of manual work to combine them into a single dataset. Due to the interference from clouds, these indexes usually also include quality bands, which indicate the quality of the index value for a given pixel, as well as missing values.\\
In comparison to MODIS, Sentinel satellite data provides a significantly higher resolution at 10 - 60 m$^2$ per pixel compared to 500-1000 m$^2$ per pixel for MODIS but there are no precalculated indexes available for Sentinel data on the Google Earth Engine platform. However, there exist scripts published by other researchers to manually calculate such indexes for example the NDVI index from Sentinel data by the Free University of Berlin~\footnote{\url{https://www.geo.fu-berlin.de/en/v/geo-it/gee/2-monitoring-ndvi-nbr/2-2-calculating-indices/ndvi-s2/index.html}, \textit{last accessed: 09.08.2023}}.\\
In comparison to MODIS and Sentinel, many LiDAR datasets are closed-source and are not available for research purposes. This is unfortunate as LiDAR data provides a very high resolution of 5-10 cm$^2$ per pixel and enables the capturing of detailed elevation data. Especially in the context of urban areas and building heights this information can be very useful, for example to calculate the sky view factor which seems to have a significant impact on air temperature modlling~\cite{dirksen2019sky}.\\
Next to index data from remote sensing, there are also other types of information that could be useful for ML applications. Alonso and Renard~\cite{alonso2020new} also used the following information:

% Todo: make into table on one page
\begin{itemize}
    \item Topographic

    \begin{itemize}
        \item Slope (°)
        \item Exposure
        \item Curvature
    \end{itemize}
    \item Land use

    \begin{itemize}
        \item Distance to railway tracks
        \item Distance to points of tourist interest
        \item Distance to subway entrances
        \item Distances to fountains
        \item Water area
    \end{itemize}
\end{itemize}

For the hyperlocal air temperature mapping study done in Oslo by Venter et al.~\cite{venter2020hyperlocal}, the following data was used:

% TODO: make into table
Hyperlocal Mapping in Oslo:
Red Landsat 7, 8 and Sentinel 2 Open source L: 30 m, S: 10 m Green Blue Near infrared Short-wave infrared 1 L: 30 m, S: 20 m Short-wave infrared 2 NDVI L: 30 m, S: 10 m IBI L: 30 m, S: 20 m Land surface temperature Landsat 7, 8 30 m Elevation above sea STRM 30 m Terrain aspect Terrain slope Terrain ruggedness CHM LiDAR Closed source 1m CHM slope CHM aspect CHM shadow/SVI Building height LiDAR + building footprint Building height sd 1–4m Building height sd 4–20 m Building height sd 20–100 m Fractional tree cover LiDAR + orthophoto Tree height Distance to coast Global water occurrence Open source 30 m Distance to fresh water\\

% TODO: more studies

% Please add the following required packages to your document preamble:
% \usepackage[table,xcdraw]{xcolor}
% If you use beamer only pass "xcolor=table" option, i.e. \documentclass[xcolor=table]{beamer}
\begin{table}[ht]
\footnotesize
\centering
\begin{tabular}{|lll|}
\hline
Measurement (Units)                                                 & Spatial Resolution & Temporal Resolution                                                                                                                    \\ \hline
\rowcolor[HTML]{EFEFEF} 
\textbf{Weather Station/Sensor Measurements}                        &                    &                                                                                                                                        \\
\begin{tabular}[c]{@{}l@{}}Air temperature (°C)\\ Mean\end{tabular} & Single location    & \begin{tabular}[c]{@{}l@{}}10 min (Sensor.Community)\\ 10 min (DWD)\\ 30 min (Netatmo Historical)\\ 10 min (Netatmo Live)\end{tabular} \\
Relative Humidity (\%)                                              & Single location    & \begin{tabular}[c]{@{}l@{}}10 min (Sensor.Community)\\ 10 min (DWD)\\ 30 min (Netatmo Historical)\\ 10 min (Netatmo Live)\end{tabular} \\
Atmospheric Pressure (mBar)                                         & Single location    & \begin{tabular}[c]{@{}l@{}}10 min (Sensor.Community)\\ 10 min (DWD)\\ 30 min (Netatmo Historical)\\ 10 min (Netatmo Live)\end{tabular} \\
Wind Strength (kmh)                                                 & Single location    & \begin{tabular}[c]{@{}l@{}}10 min (Sensor.Community)\\ 10 min (DWD)\\ 30 min (Netatmo Historical)\\ 10 min (Netatmo Live)\end{tabular} \\
Wind Direction (°)                                                  & Single location    & \begin{tabular}[c]{@{}l@{}}10 min (Sensor.Community)\\ 10 min (DWD)\\ 30 min (Netatmo Historical)\\ 10 min (Netatmo Live)\end{tabular} \\
Precipitation (mm)                                                  & Single location    & \begin{tabular}[c]{@{}l@{}}10 min (DWD)\\ 30 min (Netatmo Historical)\\ 10 min (Netatmo Live)\end{tabular}                             \\
\rowcolor[HTML]{EFEFEF} 
\textbf{Remote Sensing Data}                                        &                    &                                                                                                                                        \\
NDVI (MODIS)                                                        & 500m               & 16 days                                                                                                                                \\
EVI (MODIS)                                                         & 500m               & 16 days                                                                                                                                \\
DEM (Copernicus)                                                    & 30m                & 2015 - 2017                                                                                                                            \\ \hline
\end{tabular}
\caption{Features for Air Temperature Interpolation Used in this Work}
\label{tab:features this work}
\end{table}

\subsubsection{Features used in this Work}

Air temperature, relative humidity, atmospheric pressure, precipitation, and wind was sourced from Netatmo and Sensor.Community PWS networks as well as the DWD weather stations as reference data.
All remote sensing data acquired in this work was processed using the Google Earth Engine~\cite{gorelick2017google} as it offers a unified way of accessing data and offers anhanced processing capabilities that are especially important when dealing with these large datasets that can grow as large as several hundred terabytes. The following datasets have been used and downloaded from the Google Earth Engine platform:

\begin{itemize}
    \item MODIS/061/MOD13A1: MODIS Vegetation Indexes NDVI and EVI\\
    (500m, 16 days)~\cite{didan2021modis}
    \item COPERNICUS/DEM/GLO30: Copernicus Digital Elevations Model (30m)~\cite{copernicus30dem}
\end{itemize}

Potential datasets that could be used in the future are:

\begin{itemize}
    \item MODIS\_061\_MOD15A2H: MODIS Leaf Area Index\/FPAR \\
    (500m, 8 days)~\cite{myneni2021modis}
    \item COPERNICUS\_S2\_SR: Sentinel-2 Multi Spectral Instrument, Level-2A\\
    (10m, 5 days)~\cite{sentinel2msi} (for manual index calculation)
\end{itemize}

Additionally, location data was incoporated into the models either by longitude and latitude values in coordinate rerence system EPSG:4326 or by calculating distances between locations in meters.
The features used in this work are shown in Table~\ref{tab:features this work} and were chosen based on availability and relevance for air temperature modelling.
The amount of features in the initial scope of this work is quite limited and could be increased to gain better prediction results.

% TODO: mention python libraries used
