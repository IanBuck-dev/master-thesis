\chapter{Machine Learning-based Interpolation}
\label{chap:Machine Learning based Interpolation}

In recent times, the area of machine learning has seen big advancements in terms of model size and complexity. Especially in the area of generative \gls{ai}, transformer-based neural networks have revolutionised text and image generation. Models such as OpenAI's \textit{ChatGPT}~\cite{openai2023gpt4} or Google's \textit{LaMDA}~\cite{thoppilan2022lamda} have generated significant hype for the possibility of use of \gls{ai}\@. Additionally, statements like the universal approximation theorem, which states that a feed-forward network with a single hidden layer containing a finite number of neurons can approximate any continuous function~\cite{hornik1989multilayer}, emphasize the potential power of \gls{ml} models. As a result, the question arises what benefits \gls{ai} can bring to other areas of application, such as interpolation.\\
In this chapter, we will discuss the usage of \gls{ml} in the context of data enrichment via interpolation, more precisely in the context of Smart Cities and urban \gls{ta} interpolation. The \gls{ml} models will be compared in Section~\ref{chap:Evaluation} to traditional proven geostatistical model, e.g., Kriging, to outline and discover possible advantages and disadvantages. In general, the idea is to trade the explain- and interpretability of purely statistical-based approaches for model capabilities and accuracy, and the ability to capture more complex (non-linear) dependencies.
Due to the great flexibility of \gls{ml} models, each model can be fit to completely different use cases, such as interpolation vs.\ extrapolation or areal interpolation vs.\ interpolation of a single location. The following sections introduce different \gls{ml} models and discussed their applicability to the use-case of urban \gls{ta} interpolation.

\subsubsection{AI vs. Machine Learning vs. Deep Learning}

Before diving deeper into the applications of \gls{ml}, we need to clarify what is meant by \gls{ai}, (\gls{ml}) and \gls{dl}. \gls{ai} is a broad term that is used to describe the ability to perform tasks, that are usually associated with human intelligence. \gls{ml} is a subfield of \gls{ai}, that focuses on the ability of a system to learn from data without being explicitly programmed to do so. Finally, \gls{dl} is a subfield of \gls{ml}, that uses \gls{ann}s, or also called Simulated Neural Network (SNN), which imitate the structure of the human brain, to learn from data and perform various tasks.

\section{Machine-Learning Application Areas in Air Temperature Interpolation}

As meteorological research and analysis activities are usually in need of gridded or continuous data~\cite{sekulic2020spatio}, interpolation is a really important tool to convert single data points at distinct locations from ground-based weather stations into a continuous layer. Interpolation can also be applied to individual sensors to fill in missing data that are caused by network outages or to increase the temporal resolution to turn hourly into sub-hourly readings. Especially with the capability to increase the temporal resolution, the question arises how this capability could be combined with moving sensors to increase the spatial coverage of a sensor network. In this work, we will discuss two approaches for the use of interpolation in urban \gls{ta} sensing:

\begin{itemize}
    \item \gls{ml}-based interpolation for areas as a substitution for geostatistical methods, e.g., Kriging, to turn individual data points into a continuous grid, possibly with the ability to handle stationary and moving sensor data at the same time
    \item \gls{ml}-based interpolation for a single location to interpolate timeframes with missing data or to simulate a higher temporal resolution that does not only interpolate between individual data points of the same sensor, but also takes into consideration surrounding sensors
\end{itemize}

Research suggests that for fine-granular spatio-temporal urban \gls{ta} maps, a sensor density of at least 1 sensor per km$^{2}$ is needed~\cite{venter2020hyperlocal}, however the denser the sensor network the better the prediction quality, as even inside a single street canyon \gls{ta} can easily vary by 2 to 3°C~\cite{sugawara2008temperature}.
To achieve this sensor density as well as to gain insights into previously unobserved areas and to minimise prediction uncertainty for those areas, hybrid approaches combining stationary and moving sensors have shown to work better than purely stationary networks by covering more ground as well as reducing variability of purely mobile network setups in the context of urban temperature sensing~\cite{yang2019designing}. The combination of reference grade stationary sensors and moving sensors also shows promise in other related application, e.g., in the context of pollution island detection~\cite{iyer2022modeling}.\\
In the context of this work, we discuss both \gls{ml} applications and try to show the feasibility and potentials of \gls{ml}-based interpolation in the context of urban \gls{ta} sensing.
In the following, we introduce several \gls{ml} algorithms that can be used for regression tasks and discuss how they need to be adapted to solve interpolation tasks as well as the advantages and disadvantages of each model. The models will be implemented as prototypes and evaluated in Chapter~\ref{chap:Evaluation}.

\section{Model Selection Criteria}
\label{sec: model selection criteria}

Before using the \gls{ml} regression algorithms introduced in this section to solve the interpolation problem, the models need to be adapted to this specific use-case. This can happen either by adapting the input data and the types of features used or by adapting the model configuration. The following questions need to be answered:

\begin{itemize}
    \item \textbf{How to model sensors:} Are sensor locations modelled individually, as a network, or as a grid?
    \item \textbf{How to model the temporal correlation:} Does the model allow to model temporal correlation between sensor readings and is it only short-term or also long-term correlation?
    \item \textbf{How to model the spatial correlation:} Is spatial correlation directly incorporated in the model architecture or does it need to be modelled via features?
\end{itemize}

Next to the adaptations that need to be made to fit regression algorithms to the interpolation problem, there are also non-functional requirements that need to be considered when selecting a model. The most important requirements are:

\begin{itemize}
    \item \textbf{Model Assumptions:} The model assumptions need to be met by the features used in the input data. For example, linear regression assumes that the input features are independent from each other, as linear regression measures the amount the target variable is influenced by one feature changing while all other features stay the same. In case of correlation, this assumption is violated, as for example the amount of precipitation influences the humidity. Other assumptions could be made about the distribution of data, the mean of values, etc.
    \item \textbf{Accuracy and Reliability:} Creating an accurate and reliable model is important to increase the trust for predicted values, however there are certain trade-offs to be made, as model performance or generalisation ability are also important factors to consider.
    The accuracy of the model is mainly determined how well the chosen model can fit the underlying data, e.g., a linear model cannot fit a non-linear function, and is measured by the evaluation metrics described in Section~\ref{sec:validation methodology}.
    The reliability of the model is determined by the training data as well as the model architecture, as for example training data that is not representative of the underlying function can introduce bias into the model or can prevent the model from learning the correct function. Another important factor is the data quality, as more noise can result in worse model performance. Lastly, reliability of the model is also determined by the ability of the model to handle missing or sparse data as well as outliers. This is especially important in our context as we try to integrate moving sensors into the interpolation process, which sense data at different times and locations.
    \item \textbf{Extrapolation Capabilities:} One important factor to consider, especially when using a model with previously unseen data, is the ability to extrapolate data. Linear regression models as an example try to fit a linear function that can be easily used with data that is bigger/smaller than previously seen training data as the function is continuous. In comparison, regression trees by default do not allow for extrapolation. Also, \gls{ann}s typically perform unpredictably on unseen data.
    \item \textbf{Amount of Training Data:} The amount of training data required to train the model is another important factor to consider, as some models require more training data than others. Especially neural networks tend to need more training data than other models, as they have lots of parameters that need to be tuned. In our context, the amount of data available is quite limited, therefore models that require less training data are preferred.
    \item \textbf{Handling of Missing Data:} If the model cannot handle missing data well, there might be additional data pre-processing steps that need to be done. One example for this would be how the model reacts to \gls{nan} values which is a float number defined in the IEEE 754 floating-point standard~\cite{ieee754}. Each multiplication with \gls{nan} results in \gls{nan}, which can therefore lead to a model where all weights turn into \gls{nan} when there is a single \gls{nan} value in the input data. This is especially important in the context of distributed sensors, as they might not sense every feature at all times. Common strategies to handle missing values involve dropping the complete feature if it has any \gls{nan}, drop any rows in the data that has \gls{nan} values or imputation, e.g., replace the missing value with a value such as the mean or median of the feature.
    \item \textbf{Handling of Sparse Data:} Like missing data, handling of sparse data is also important to prevent problems such as the \gls{nan} problem mentioned previously. The main difference to missing data is that sparse data is not missing, but rather not always available. In our context, moving sensors would be an example for sparse data, as they only sense data at certain times and locations. The strategies to handle missing data are also similar to those of missing data, but imputation and interpolation are more common strategies to handle sparse data.
    \item \textbf{Model Performance:} The more complex a model is, the more training data is needed to fine-tune all its weights and the longer it takes to train, either due to the amount of data or the number of steps that need to be taken when updating weights in the training process. In the context of open-source and citizen participation less complex models are preferred, as they can be trained and deployed with less resources. However, a less complex model could have the downside of not being able to fit the underlying data as well as a more complex model, therefore there is a trade-off between model complexity and model performance. Another factor is the capability of handling many locations and data points, as some models' performance degrades significantly when the amount of data points increases, as well as the ability to handle high-dimensional data.
    \item \textbf{Model Capabilities vs. Interpretability:} A common trade-off in \gls{ml} models is between model capabilities and complexity and interpretability of the model as done by~\cite{zumwald2021mapping}, as `Neural network and deep learning approaches allow for large flexibility and predictive power but are harder to interpret than ensemble-based approaches which allow for the required flexibility, while still providing insights into the algorithms' inner workings, e.g., via variable importance and prediction uncertainty estimation'.
    \item \textbf{Other:} Next to these main requirements, there are also other requirements, such as the ability the handle massive amounts of data, live retraining, or sophisticated support via \gls{ml} libraries such as \textit{Tensor Flow}~\footnote{\url{https://www.tensorflow.org/}} for commercial use cases. Due to the limited scope of this work, these requirements will be considered in less detail.
\end{itemize}

After introducing the requirements for model adaptions and selection, the next step is to introduce and compare the different \gls{ml} regression algorithms.
Generally speaking, \gls{ml} algorithms can be categorised based on many different properties~\cite{sarker2021machine}, such as the type of learning, e.g., supervised, unsupervised, semi-supervised or reinforcement, or the type of problem they try to solve, e.g., classification, regression or clustering. The most important differentiation for this work is to distinguish algorithms based on the type of problem they try to solve. Because we focus on solving interpolation problems, in this work we will only consider algorithms that can be used to solve regression problems, as interpolation is a form of regression.\\
In regression analysis, the goal is to predict a (continuous) target variable $y$ based on a set of input variables $X$, like it is the case for temperature interpolation. This problem can be classified as a supervised learning problem, therefore possible algorithm candidates contain the following models:

\begin{itemize}
    \item (Multi) Linear Regression
    \item K-Nearest Neighbours Regression
    \item Regression Trees and Forests
    \item Histogram-Based Gradient Boosting
    \item Support Vector Regression
    \item (Deep) Neural Networks
\end{itemize}

Next to these algorithms, there also exist less popular interpolation methods, such as outlined in~\cite{li2014spatial}, however due to the limited scope of this work, we only take a look at \gls{ml} models listed above, who are implemented in the popular \textit{sklearn} \gls{ml} library~\cite{scikit-learn}.
Each model has certain benefits but also comes with drawbacks or special assumptions for the input data to the model. First, we will discuss each model and then compare these assumptions with the data coming from the data-layer, to identify suitable models for the task of \gls{ta} interpolation.
Based on the domain, there already exist proposed best practices for which algorithms to use for what applications. In the context of Smart Cities such recommendations exist for topics such as intelligent transportation systems, smart grids, smart city health care and more can be found in~\cite{ullah2020applications}, which does not cover interpolation. In the context of \gls{ta} interpolation, regression forests and \gls{hgb} seem to be popular choices and perform better compared to other methods~\cite{apaydin2022evaluation, ho2014mapping}.

\section{Comparison of Machine Learning Algorithms}
\label{sec:comparison ml algorithms}

\subsection{(Multi) Linear Regression}
\label{subsec: linear regression}

Linear regression~\cite{montgomery2021introduction} is a comparatively simple, yet very powerful and widely used model for regression problems. The goal of this model is to predict a continuous dependent variable based on several independent variables. These independent variables can be either continuous or discrete. The model can be expressed as follows:

\begin{equation}
    y = \beta_0 + \beta_1 x_1 + \beta_2 x_2 + ... + \beta_n x_n + \epsilon
\end{equation}

where $y$ is the dependent variable, $x_1$ to $x_n$ are the independent variables, $\beta_0$ to $\beta_n$ are the parameters of the model and $\epsilon$ is the error term. The relationship between the parameters is assumed to be linear, while each variable must be independent of each other. Due to this independent assumption, there are special steps needed to make linear regression work for (geo-) spatial data, as these types of variables are usually correlated with each other. This is further discussed in~\ref{sec: additional considerations}.\\
Linear regression has the advantage that it is a very simple model, therefore most of the work needs to be done in the feature engineering process. The downside is that there is no inherent support for spatial or temporal correlation and the model cannot be used to fit non-linear functions. To fit non-linear functions, polynomial regression can be used, which can however lead to overfitting especially when the degree of the polynomial is high.\\
Linear models seem to perform worse in urban temperature related settings, as they are unable to capture non-linear effects~\cite{voelkel2017towards}, therefore this model with only be evaluated briefly.

\subsection{KNN Regression}

\gls{knn} is a simple algorithm that can be used for both classification~\cite{cover1967nearest} and regression problems~\cite{altman1992introduction}. The main idea behind \gls{knn} is the assumption, that data points near each other are more similar than data points that are further away. As a result, the $k$ nearest neighbours, either by number or by radius, of a data point are used to predict the target variable. The number of nearest neighbours is a hyperparameter that needs to be tuned to find the best trade of between bias and variance. The model with equal weights per neighbour can be expressed as follows:

\begin{equation}
    \hat{y} = \frac{1}{k} \sum_{i=1}^{k} y_i
\end{equation}

where $\hat{y}$ is the predicted target variable, $k$ is the number of nearest neighbours and $y_i$ is the target variable of the $i$-th nearest neighbour. Additionally, weight could be assigned to individual neighbours to for example weight closer neighbours higher.\\
\gls{knn} is part of the family of non-parametric models, meaning they do not make any strong assumptions about the underlying regression curve. \gls{knn} is a simple yet powerful model and based on the weight function, all predictions can either be weighted equally, by distance, or by a custom function, that allows potentially more complex weight calculations.

\subsection{Regression Trees and Random Forests}

Tree predictors are used for a wide variety of classification and regression problems. Since tree predictors are unstable, e.g., vary significantly given similar inputs, and tend to overfit, random forests were introduced as a counter measure. \gls{rf}s combine multiple tree predictors and train them on different features and subsets of data and either average their predictions to reduce the variance or use boosting methods to reduce the bias of a combined estimator~\cite{breiman2001random}.\\
In the case of predicting a continuous target variable, regression trees can be used. The principle behind regression trees is to split the data into continuously smaller sub-sets and organise the splitting points in a way that minimises the error. Compared to decision trees which try to minimise the entropy, regression trees try to minimise an error that is compatible with a continuous target variable such as \gls{mse}.\\
Regression trees are comparatively easy to understand and interpret and offer certain benefits such as feature insensitivity, meaning that features do not need to be scaled before usage and can be used as is. In~\cite{zumwald2021mapping} Zumwald et al.\ choose to use \gls{qrf}~\cite{meinshausen2006quantile} for mapping hyperlocal \gls{ta} in Zurich, Switzerland, due to the flexibility and predictive power of ensemble-based approaches and the ability to still gain additional insights into the algorithms' inner workings via variable importance and prediction uncertainty estimation. They note however that \gls{ann} and \gls{dl} approaches allow for even larger flexibility and predictive power but lack behind in other areas such as the aforementioned interpretability. In the following we get an overview of \gls{rmse} and \gls{r2} values achieved by related studies.\\
Ho et al.~\cite{ho2014mapping} mapped maximum daily \gls{ta} for Vancouver, Canada on hot summer days and combined remote sensing TM/ETM data from Landsat with field observations from Environment Canada and \gls{wow}. They compared Ordinary least squares regression, \gls{svm} regression, and \gls{rf} Regression and achieved the following \gls{rmse}:

\begin{itemize}
    \item \gls{rf}:\ \gls{rmse} 2.31°C
    \item \gls{svm}:\ \gls{rmse} 2.46°C
    \item Ordinary least squares regression:\ \gls{rmse} 2.46°C
\end{itemize}

They also added that stations closer to the ocean had higher estimation errors, possibly due to more variable wind patterns~\cite{runnalls2000dynamics}.\\
Hengl et al.~\cite{hengl2018random} adapted the \gls{rf} algorithm to take advantage of spatial and temporal variables, called \gls{rfsp}. They compared the following ways of modelling spatial relations:

\begin{itemize}
    \item Geographical coordinates (north, east)
    \item Euclidean distances to reference points (centre, edges) of the study area 
    \item Euclidean distances to sampling locations, e.g., distance matrices also used in geostatistics
    \item Downslope distances for hydrological analysis
    \item Resistance distances or weighted buffer distances
\end{itemize}

They found that only using coordinates resulted in lower \gls{r2} and higher \gls{mse} errors compared to distances between sampling locations, and concluded that \gls{rfsp} performed similar to \gls{ok}, however without the downsides of needing to create and fit a variogram, no manual trend building, no need to define a search radius for Kriging, no transformation of the target variable, spatial autocorrelation and correlation with spatial environmental factors are modelled together, and variable importance shows which observations and covariants have the most influence on the predictions. They described the advantages of \gls{rf} as follows:

\begin{itemize}
    \item Information overlap (multicollinearity) and over-parametrization are not a problem. \gls{rf} can even be trained with more covariants than observations.
    \item \gls{rf} has built-in bagging methods to select subsets of features and training samples to reduce bias. In geostatistics over-parametrization and overlap of covariants are an issue and can lead to bias predictions.
    \item noise-resistance~\cite{strobl2007bias}
    \item Distance measures can be extended to more complex calculations
\end{itemize}

The downsides of \gls{rf} according to them include:

\begin{itemize}
    \item Extrapolation can lead to even poorer results compared to linear models.
    \item If the training data is biased, the \gls{rf} will have biased predictions. Therefore, stricter cross-validation techniques might be necessary.
    \item Model size is much larger compared to linear models.
    \item Model is optimized for representing the training data, not the spatial and spatiotemporal dependence structure.
    \item Computationally expensive to train predictor and make predictions.
    \item Distance matrices scale quadratically, therefore higher number of points and greater radiuses/number of neighbours result in way higher computational cost. 
\end{itemize}

Lastly, they also concluded, that the reduction of trees in the forest from 500 to 100 often didn't result in a big loss in accuracy, suggesting that simpler \gls{rf} models can retain high levels of accuracy.\\
\textit{Sekuli{\'c} et al.} further improved \gls{rfsp} and introduced \gls{rfsi}~\cite{sekulic2020random} by including nearest observations and distances as covariants. They compared their model performance against \gls{ok}, \gls{rfsp}, \gls{idw}, NN, and trend surface mapping~\cite{chorley1965trend} to compare the performance across one synthetic dataset, one precipitation dataset, and one mean daily temperature dataset for Croatia 2008 with a resolution of 1km$^2$. In the real case-studies. \gls{rfsi} outperformed the other methods with a \gls{rmse} in the mean daily temperature mapping case-study of 1.4 compared to \gls{rf} and \gls{rfsp} with \gls{rmse} 1.6, \gls{idw} with 1.8 and Space-Time Regression Kriging with 2.4. The covariants to model daily mean temperature were latitude, longitude, distance-to-coastline, elevation, seasonal fluctuation, insolation (total incoming solar radiation), and \gls{modis} \gls{lst}. They found that `Seasonal fluctuation, \gls{modis} \gls{lst} images, insolation, and distance-to-coastline were the most important covariates for \gls{rf} and \gls{rfsp}', while in \gls{rfsi} the observations from ground stations as covariants were the most important.\\
In~\cite{venter2020hyperlocal} \textit{Venter et al.} mapped \gls{ta} over Oslo, Norway in 2018 on a scale of 10-30m using Sentinel, Landsat, LiDAR data which they combine with Netatmo \gls{pws} data from 1310 stations. They achieve an average \gls{rmse} of 0.52 °C (\gls{r2} = 0.5), 1.85 °C (\gls{r2} = 0.05) and 1.46 °C (\gls{r2} = 0.33) for annual mean, daily maximum, and minimum \gls{ta}. The `models performed best outside of summer months when the spatial variation in temperatures were low and wind velocities were high'.\\
Similarly, \textit{Zumwald et al.} mapped hourly \gls{ta} on a 10m$^2$ grid during a heat wave from 25-29th June 2021 in Zurich (Switzerland) and compared reference sensors, AWEL sensors and \gls{pws} data from Netatmo stations and achieved \gls{rmse}s of 1.43°C, 1.35°C, and 1.71°C for the respective types~\cite{zumwald2021mapping}. They used \gls{qrf} as their \gls{ml} model of choice. Important features in their work were \gls{ta} at 2m (33\%), \gls{ta} at 5cm (21\%), relative humidity at 2m (13\%), global radiation (8\%), and more.\\
Both~\cite{zumwald2021mapping, zhang2012bias} found that there seems to be a systematic bias when estimating \gls{ta} using \gls{rf} to underestimate high temperatures and overestimate low temperatures.

\subsection{Histogram-Based Gradient Boosting}

Similar to \gls{rf}s, \gls{hgb} is an ensemble-based estimator that combines multiple estimators, in this case gradient boosting decision trees, and averages the results to get a more robust estimation. Compared to \gls{rf}s, \gls{hgb} as implemented by sklearn~\footnote{\url{https://scikit-learn.org/stable/modules/generated/sklearn.ensemble.HistGradientBoostingRegressor.html}, \textit{last accessed: 25.08.2023}} based on LightGBM~\cite{ke2017lightgbm}, should have better performance, especially for bigger datasets and higher dimensional features, has built-in support for missing values which simplifies data pre-processing steps, and finally seems to slightly outperform \gls{rf} according to the sklearn documentation~\footnote{\url{https://scikit-learn.org/stable/auto\_examples/ensemble/plot\_forest\_hist\_grad\_boosting\_comparison.html}} and other studies~\cite{apaydin2022evaluation}.

\subsubsection{Bagging Methods}

To decrease the variance of a single tree predictor, bagging is used to introduce randomization into the training process. There are several different bagging methods:

\begin{itemize}
    \item \textbf{Pasting}: Splitting the data into different subsets and training a tree predictor on each subset~\cite{breiman1999pasting}
    \item \textbf{Bagging}: Splitting the data into different subsets but with replacement~\cite{breiman1996bagging}
    \item \textbf{Random Subspaces}: Splitting the data into different subsets of features~\cite{ho1998random}
    \item \textbf{Random Patches}: Splitting both samples and features into different subsets~\cite{louppe2012ensembles}
\end{itemize}

\gls{rf} and \gls{hgb} already use built-in bagging methods, so we do not consider these bagging methods further in this work.

\subsection{Support Vector Machine Regression}

\gls{svm}s are typically used in classification problems however they can be also used for regression problems, called \gls{svr}. \gls{svm}s transform the input data into a higher dimensional space and try to find a hyperplane that separates the data into two classes. This approach is similar to linear regression, however \gls{svm}s are more robust to outliers and can be used for non-linear problems. Depending on the Kernel function used, e.g., Linear, Polynomial, \gls{rbf}, or Sigmoid, the model can suffer from the same problems as linear regression when dealing with correlated spatial data.
As a result, appropriate counter measures need to be taken, such as using the Mahalanobis distance instead of the Euclidean distance for \gls{rbf} kernels~\cite{kamada2006support}, which converts correlated features into uncorrelated features.

\subsection{Neural Networks}

\gls{ann}s are a more advanced \gls{ml} method that takes inspiration from the human brain and electrical impulses being transmitted by neurons. An \gls{ann} is build-up of neurons which are grouped in layers that are connected and have activation functions and weights, that get trained during the learning process. The simplest \gls{ann} is the perceptron~\cite{rosenblatt1957perceptron} which models one single neuron, consisting of one or many inputs, a single processor, and a single output. An \gls{ann} usually consists out of one input and output layer, and one to many hidden layers. If there are more than one hidden layer, the \gls{ann} is usually called Deep Neural Network (DNN) and we speak from Deep Learning~\cite{lecun2015deep}. \gls{ann}s have many hyperparameters and weights to train, therefore they need more data to train. Additionally, \gls{ann}s act more like a black box, so model analysis such as feature importance is not possible. Using \gls{ann}s is a trade-off between model capability, available data to train and test, and the explainability of the model.\\
As previously mentioned, the universal approximation theorem by Hornik et al.\ states how capable \gls{ann}s can be~\cite{hornik1989multilayer}. Depending on how the model is setup, e.g., as a feed-forward network or a directed acyclic graph where layers feed back into each other, the type of loss-function, and the type of learning method, such as Stochastic Gradient Descend, \gls{ann}s can be adapted to many problems such as regression and interpolation.\\
Due to the complexity of \gls{ann}s and the lacking explainability of the model, we do not focus on this \gls{ml} method in this work however want to keep in mind that \gls{ann}s could be a powerful tool for tasks such as \gls{lst} and \gls{ta} estimation and prediction~\cite{yuan2020deep}.
