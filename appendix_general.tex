\chapter{Appendix}

\section{Netatmo}
\label{appendix: netatmo}

\begin{figure}[ht]
    \centering
    \includegraphics[width=.8\textwidth]{images/qc_hamburg_netatmo_june_before_after.png}
    \caption{QC for Netatmo in Hamburg showing difference between day and night, 19.06.2023}
    \label{fig:qc netatmo hamburg before after m5}
\end{figure}

\begin{figure}[ht]
    \centering
    \includegraphics[width=0.9\textwidth]{images/eval imputed vs not imputed.png}
    \caption{Evaluation Single Station Interpolation, Detailed View}
    \label{fig:eval single station interpolation detailed}
\end{figure}

\begin{figure}[ht]
    \centering
    \includegraphics[width=0.9\textwidth]{images/eval imputed vs not imputed part 2.png}
    \caption{Evaluation Single Station Interpolation, Detailed View}
    \label{fig:eval single station interpolation detailed part 2}
\end{figure}

\section{Sklearn Model Parameters for Single Station Interpolation}
\label{appendix: sklearn ml parameters single station}

\begin{lstlisting}[language=Python, caption=Random Forest Regressor Parameters]
class sklearn.ensemble.RandomForestRegressor(n_estimators=200,
*, criterion='squared_error', max_depth=None, min_samples_split=2,
min_samples_leaf=1, min_weight_fraction_leaf=0.0, max_features=1.0,
max_leaf_nodes=None, min_impurity_decrease=0.0, bootstrap=True,
oob_score=False, n_jobs=None, random_state=None, verbose=0,
warm_start=False, ccp_alpha=0.0, max_samples=None)
\end{lstlisting}

The number of trees was increaed from 100 to 200 to increase the performance.

\begin{lstlisting}[language=Python, caption=Histogram-based Gradient Boosting Parameters]
class sklearn.ensemble.HistGradientBoostingRegressor
(loss='squared_error', *, quantile=None, learning_rate=0.1,
max_iter=200, max_leaf_nodes=31, max_depth=None, min_samples_leaf=20,
l2_regularization=0.0, max_bins=255, categorical_features=None,
monotonic_cst=None, interaction_cst=None, warm_start=False,
early_stopping='auto', scoring='loss', validation_fraction=0.1,
n_iter_no_change=10, tol=1e-07, verbose=0, random_state=42)
\end{lstlisting}

The number of max iterations was increased from the default value of 100 to 200 to improve the performance of the model and the random state was set to 42 so all interations yield the same result.

\section{Feature Engineering}

\begin{lstlisting}[language=Python, caption=Timestamp to Sinus Curve Feature, label=lst: timestamp to sin]
    # Convert datetime64 timestamp hours and minutes to circle angle
    filtered_gdf['time'] = filtered_gdf['time'].apply(lambda x: (x.hour * 60 + x.minute) * 2 * np.pi / (24 * 60))
    filtered_gdf.rename(columns={'time': 'time_angle'}, inplace=True)

    # Calculate continuous representation of time as an angle in radians
    filtered_gdf['sin_time'] = np.sin(filtered_gdf['time_angle'])
\end{lstlisting}

\section{Histogram-based Gradient Boosting Single Location Interpolation}

\subsubsection{List of stations for minimum distance between stations}
\label{appendix stations for minimum distance between stations}

The list of stations for the minimum distance between stations is the following by their Netatmo station id:

\begin{itemize}
    \item 70:ee:50:83:b1:e2 
    \item 70:ee:50:16:16:ce
    \item 70:ee:50:5e:d4:16
    \item 70:ee:50:00:d3:96
    \item 70:ee:50:6b:5f:50
    \item 70:ee:50:00:d1:1c
    \item 70:ee:50:5f:51:a0
    \item 70:ee:50:6b:97:86
    \item 70:ee:50:28:f2:ca
    \item 70:ee:50:58:e8:70
\end{itemize}

The locations of the stations are presented in Figure~\ref{fig:eval_hamburg_locations_point_histb_10_map}.

\begin{figure}[ht]
    \centering
    \includegraphics[width=1\textwidth]{images/eval_hamburg_locations_point_histb_10_map.png}
    \caption{Netatmo Stations for Minimum Distance Between Stations, Hamburg}
    \label{fig:eval_hamburg_locations_point_histb_10_map}
\end{figure}

\section{Areal Interpolation}

Figure~\ref{fig:dwd stuttgart june 23 mean min max} shows the mean/max/min air temperature for DWD station with id 4931 across June 2023 in Stuttgart near the airport.
Figure~\ref{fig:dwd stuttgart june 23 10 min 2m 5cm} shows the air temperature for the same station on the 19. June 2023 in a 10 minute interval. The blue line, i.e., TT\_10, shows the air temperature at 2m while the orange line, i.e., TM5\_10, shows the air temperature at 5cm directly above the ground.

\begin{figure}[ht]
    \centering
    \includegraphics[width=1\textwidth]{images/dwd_stuttgart_june_23_tair_max_mean_min.png}
    \caption{DWD Station 4931, Stuttgart, June 2023}
    \label{fig:dwd stuttgart june 23 mean min max}

    \includegraphics[width=1\textwidth]{images/dwd_stuttgart_june_19_23_tair.png}
    \caption{DWD Station 4931, Stuttgart, 19. June 2023}
    \label{fig:dwd stuttgart june 23 10 min 2m 5cm}
\end{figure}
