\chapter{Related Work}
\label{chap:Related Work}

% start: Heat island detection -> needs temperature map of current climatic conditions of the intra urban micro climate
% - LST data (high coverage, but low spatialtemporal resolution)
% - coverage through public stationary weather stations (low spatial resolution)
% - option: sensor network with many stationary + moving sensors
% (hybrid approaches show promising higher prediction quality)

% but sensor networks have potentially low spatial coverages.
% -> solution: interpolate missing data (calculation or educated guess)

% First overview: interpolation techniques/history
% - foundation in statistics
% - adaptation for specific research areas (focus on temperature related research in this thesis)
%     - geostatistics (Krigin) -> probabilities need to be known
%         - climate research (macro) -> global warming, temperature trends
%         - urban climate research (micro) -> heat islands, polution
%     - ML approaches (deep learning, random forests etc.)
%         - list some approaches

% ML show promise in helping with interpolation tasks
% -> in order to train we need good data sets

% overview of data collection
% - open data movement
%     - list open data catalogues (EU, more local ones)
% - commercial sources
%     - list some examples, but not focus because not accessible
% - overview data preperation
% - overview of feature selection
% - overview ML interpolation methods

In the following chapter we lay the foundation for the research conducated in this thesis.

\section{Interpolation of Missing Data}


\subsection{Regression Analysis in Statistics}
- foundation of other research fields, based in statistics/mathematics

- linear regression (least-squares)
- multiple regression models
- hierachical regression

- special cases
    - piecewise linear regression
    - inverse prediction
    - weighted least squares
    - logistic regression
    - poisson regression

\subsection{Interpolation in Geostatistics}

- spatiotemporal (kriging)
- time series prediction vs interpolation of missing data
- based on GIS 
- pipeline: fit measured data points to grid, interpolate missing squares


\subsection{Prediction of Future Data}
- Time Series Analysis

\subsection{Interpolation with Machine Learning Models}

- ML regression

\section{Access to Data-Sets}

\subsection{OpenData Movement}
 - portals
        - official:
            - EU: https://data.europa.eu/data/datasets?query=temperature\&locale=en (combinas many governmental and local catalogues) / https://data.europa.eu/data/catalogues?locale=en
            - USA: https://data.gov/
            - UK: https://www.data.gov.uk/
        - private: https://www.kaggle.com/

- strategies:
    - self procurement (test beds -> Helsinki Testbed (climate research meso sclae), UK Birmingham Testbed (climate + smart city), )

\section{Applications and Research Areas}

- focus on temperature interpolation/climate research

- climate research
    - high area coverage, low spatio and temporal resolution (5km by 5km squares)
    - based on LST data (from satelites) -> not the same as air temperature

- micro-climate research
    - bad/costly area coverage, high spatio and temporal resolution
    - Urban heat islands
    - Pollution (fine dust pollution)

- connection to smart cities
    - integrating many heterogenous data points
    - detection climatic anomalys
    - notifiy/warn residents

%% Important keys
- important key words

- current status quo

- important authors and current work

-> identify reseach gap
- convert single data points to continous/gridded data
- improve density of data to gain insights and imrove visibility -> identify areas with low prediction quality
